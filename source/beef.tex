\documentclass[a4paper]{article}

%\usepackage{CJKutf8} 
\usepackage{somesty}
\usepackage{palatino}
\usepackage{booktabs}
\usepackage{multirow}
\usepackage{subfigure}
\usepackage{graphicx}           %图片插入宏包
\usepackage{caption}


\setlength\parskip{.8\baselineskip}
\setcounter{page}{1}
\pagestyle{fancy}
\lhead{}
\chead{}
\rhead{\small \thepage }
\lfoot{}
\cfoot{}
\rfoot{}

\title{}
\author{}
\date{}

\begin{document}
\maketitle
\textsf{\textsl{The first article in this unit sets out the vegetarian's argument against eating meat. Here, for those of you who still want to carry on eating meat with an easy heart, we have the meat eater's reply.}}

\begin{center}
\textbf{ \huge Where's the Beef?}
\end{center}


\begin{flushright}
\textsf{\textit{Alan Herscovici}}
\end{flushright}

With summer comes that most wonderful of North American traditions, the backyard barbecue. The succulent aroma of fresh grilled steak, sausages, chicken and fish draws family, friends and neighbours together for a communal feast. Inevitably, in these politically correct times the conversation may drift to the question of whether we really ought to be eating meat at all.

The following guide should help see you through until the burgers are done.

Appealing to self-interest, a common opening line for proselytizing vegetarians is to claim that "eating meat is bad for us." They have trouble explaining, however, why human health and longevity have improved steadily as animal products became more readily available throughout this century. In fact, meat is an excellent source of 12 essential nutrients, including protein, iron, zinc and B vitamins.

 It is true that excessive fats can be harmful, but today's meats are lean. Based on equal-size servings, tofu has more fat than a sirloin steak and only half the protein. (Tofu also makes a mess of the grill.)

With the exception of certain religious sects, people have rarely been vegetarian by choice. Most often, vegetarianism is the unfortunate result of poverty. Yet the veggie crowd also claims that "humans are not natural meat-eaters." Our teeth are not as sharp and our intestinal tracts not as short as those of cats and other pure carnivores. But we are not equipped to be herbivores, either. Like other omnivores (such as bears or racoons), our digestive equipment allows us to tackle a wide range of foods.

If we were not designed to eat meat, why do we produce large quantities of the enzymes required to break down such foods? Why is vitamin B12 (found only in animal products) essential to human life? If we were not natural meat-eaters, or at least bug and grub eaters, our species would have died out long ago. If we did not develop as hunters, why are our eyes in the front to our heads like those of other predators (tigers, wolves or owls)? Why does the mere smell of a sizzling steak set my saliva glands watering?

Shifting their ground, animal activists now charge that livestock threatens the environment. But much of the world's arable land is best suited to be used as pasture. It is too hilly, fragile, dry or cold for cultivation. Cattle convert grass into nutrients that can be digested by humans. Those who promote organic agriculture understand that livestock completes the nutrient cycle by returning organic matter to the soil with manure.

 Other anti-meat myths can also be dismissed. For example: 

 Whatever you may think about fast food hamburgers, eating them does not encourage the destruction of Amazon rainforests. Because of disease-control measures, no unprocessed South American beef products at all may be imported into Canada.

 Livestock do not use up grains that could otherwise feed starving people in Third World countries. The main diet of cattle is grass and hay. Pigs, chickens and other farm animals are generally fed corn and barley, while people eat mainly wheat and rice. Animals also consume pest-and weather-damaged grains, crop residues (corn stalks and leaves) and by-products from food processing, such as unusable grains (or parts of grains) left over from producing breakfast cereals and other human foods. Raising livestock in Canada does not prevent us from shipping emergency supplies to people in need. Hunger today, however, is usually the result of political, economic and distribution problems, not a lack of production capacity. 

The production of methane gas by livestock is not a major contributor to global warming. Methane gas is only one of many possible "greenhouse" gases. It is produced by all sorts of decomposition of organic matter, including normal digestion (even by vegetarians). Main sources of greenhouse gases include wetlands, forest fires, landfills, rice paddies, the extraction of gas, oil and coal---and even termites. 

 Meat does not contain harmful pesticide, antibiotic or other residues. This is assured by stringent Agriculture Canada and Health Canada regulations and inspection. Concerns about dangerous bacteria are easily addressed by cooking your meat well. (Fruit and raw vegetables, in fact, present a more difficult problem.)

One study that is not often cited by animal activists is a recent report by the Centre for Energy and the Environment at the University of Exeter in England. David Coley and his associates analyzed how much fuel energy is used to produce and process different foods. Burning fuel releases carbon into the atmosphere, the major suspected cause of global warming.

 To the dismay of the politically correct set, meat scores far better than vegetables on this environmental-impact scale. It requires eight megajoules of fuel energy to produce enough beef or burgers to provide one megajoule of food energy. The fuel energy costs of chicken and lamb are seven megajoules and six megajoules respectively. Typical salad vegetables, however, require as much as 45 megajoules of fuel energy for each energy unit of food intake provided.

"Meat does well because it is not highly processed, provides a lot of calories and is often produced locally." Coley reported in New Scientist last December.

It would require more ink than is available to us here to respond to all the claims animal activists have made about the supposed evils of modern livestock husbandry methods, what they misleadingly label "factory farming." For example, they criticize the caging of laying hens, while ignoring the fact that such systems improve hygiene, preventing disease and reducing the need for antibiotics.

Detailed responses to animal-welfare concerns are provided in Food for Thought: Facts about Food and Farming, published by the Ontario Farm Animal Council.

For debate around the barbecue, suffice it to say that animals cannot be productive unless they receive excellent nutrition and care. Farmers who do not provide good care for their animals will not remain in business for long.

 Once fallacious claims about health, environment and animal welfare are stripped away, the heart of the animal-rights argument is exposed. What right, they ask, do we have to use animals at all?

The central fallacy of this argument is that it ignores basic principles of biology and ecology. Every plant and animal species naturally produces far more offspring than their environment can support to maturity. This "surplus" provides food for other species. Aboriginal people called this "the cycle of life." We now usually call it "the food chain." We are part of this cycle, like every other living organism on the planet. The domestication of livestock has been a very successful survival strategy, not only for humans, but also for the other species involved.

The squeamishness some people now feel about eating animals does not represent a more evolved sensitivity to nature. It is a symptom of how cut off some people have become from nature.

Thanks to modern agriculture, many city people now take our abundant food supply for granted. We forget that all our food must still be wrested from the land. Even our vegetables must be protected from other creatures. Even a carrot clings to the soil with all its strength. Like other animals, we kill to eat. But because we are human, we can also give thanks and treat the animals that feed us with respect.

I think those burgers should be ready about now...

(1222 words)

\bibliography{}
\bibliographystyle{plain}
\end{document}
    
