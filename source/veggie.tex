\documentclass[a4paper]{article}

%\usepackage{CJKutf8} 
\usepackage{somesty}
\usepackage{palatino}
\usepackage{booktabs}
\usepackage{multirow}
\usepackage{subfigure}
\usepackage{graphicx}           %图片插入宏包
\usepackage{caption}


\setlength\parskip{.8\baselineskip}
\setcounter{page}{1}
\pagestyle{fancy}
\lhead{}
\chead{}
\rhead{\small \thepage }
\lfoot{}
\cfoot{}
\rfoot{}

\title{}
\author{}
\date{}

\begin{document}
\maketitle
\textsf{\textsl{ Picture in your mind all the dishes that go to make up your ideal meal. Now take away all the dishes with meat in them. Not much left on the table?  Then Joseph Pace has a word or two to say to you in an attempt to get you to change your eating habits.}}

\begin{center}
    \textbf{\huge Let's Go Veggie!}
\end{center}

\begin{flushright}
    \textsf{\textit{Joseph Pace}}
\end{flushright}

If there was a single act that would improve your health, cut your risk of food-borne illnesses, and help preserve the environment and the welfare of millions of animals, would you do it?

The act I'm referring to is the choice you make every time you sit down to a meal.


More than a million Canadians have already acted: They have chosen to not eat meat. And the pace of change has been dramatic.

Vegetarian food sales are showing unparalleled growth. Especially popular are meat-free burgers and hot dogs, and the plant-based cuisines of India, China, Mexico, Italy and Japan.

Fuelling the shift toward vegetarianism have been the health recommendations of medical research. Study after study has uncovered the same basic truth: Plant foods lower your risk of chronic disease; animal foods increase it.

The American Dietetic Association says: "Scientific data suggest positive relationships between a vegetarian diet and reduced risk for several chronic degenerative diseases."

This past fall, after reviewing 4,500 studies on diet and cancer, the World Cancer Research Fund flatly stated: "We've been running the human biological engine on the wrong fuel."

This "wrong fuel" has helped boost the cost of degenerative disease in Canada to an estimated \$400 billion a year, according to Bruce Holub, a professor of nutritional science at the University of Guelph.

Animal foods have serious nutritional drawbacks: They are devoid of fiber, contain far too much saturated fat and cholesterol, and may even carry traces of hormones, steroids and antibiotics. It makes little difference whether you eat beef, pork, chicken or fish.

Animal foods are also gaining notoriety as breeding grounds for E. coli, campylobacter and other bacteria that cause illness. According to the Canadian Food Inspection Agency, six out of ten chickens are infected with salmonella. It's like playing Russian roulette with your health.

So why aren't governments doing anything about this? Unfortunately, they have bowed to pressure from powerful lobby groups such as the Beef Information Center, the Canadian Egg Marketing Agency and the Dairy Farmers of Canada. According to documents retrieved through the Freedom of Information Act, these groups forced changes to Canada's latest food guide before it was released in 1993.

This should come as no surprise: Even a minor reduction in recommended intakes of animal protein could cost these industries billions of dollars a year.

While health and food safety are compelling reasons for choosing a vegetarian lifestyle, there are also larger issues to consider. Animal-based agriculture is one of the most environmentally destructive industries on the face of the Earth.

Think for a moment about the vast resources required to raise, feed, shelter, transport, process and package the 500 million Canadian farm animals slaughtered each year. Water and energy are used at every step of the way. Alberta Agriculture calculates that it takes 10 to 20 times more energy to produce meat than to produce grain.

Less than a quarter of our agricultural land is used to feed people directly. The rest is devoted to grazing and growing food for animals. Eco-systems of forest, wetland and grassland have been decimated to fuel the demand for land. Using so much land heightens topsoil loss, the use of harsh fertilizers and pesticides, and the need for irrigation water from dammed river. If people can shift away from meat, much of this land could be converted back to wilderness.

The problem is that animals are inefficient at converting plants to edible flesh. It takes, for example, 8.4 kilograms of grain to produce one  kilogram of pork, the U.S. government estimates.

After putting so many resources into animals, what do we get out? Manure --- at a rate of over 10,000 kilograms per second in Canada alone, according to the government. Environment Canada says cattle excrete 40 kilograms of manure for every kilogram of edible beef. A large egg factory can produce 50 to 100 tonnes of waste per week, the Ontario Ministry of Agriculture estimates.

And where does it go? In the 1992 Ontario Groundwater Survey, 43 per cent of tested wells were contaminated with agricultural run-off containing fecal coliform bacteria and nitrates. Earlier this month, charges were laid against a large Alberta feedlot operator for dumping 30 million litres of cattle manure into the Bow River, "killing everything in its path," as a news story described it.

And then there is methane, a primary contributing gas in global warming and ozone layer depletion. Excluding natural sources, 27 per cent of Canada's and 20 per cent of the world's methane comes from livestock.

John Robbins, author of the Pulitzer prize-nominated book Diet for a New America (Group West), said it best when he stated: "Eating lower on the food chain is perhaps the most potent single act we can take to halt the destruction of our environment and preserve our natural resources."

Our environment also includes the animals killed for their meat. It has become an accepted fact that today's factory-farmed animals live short, miserable, unnatural lives.

As part of my research at the University of Waterloo, I toured some of the country's largest "processing" plants. The experience has left me with recurring nightmares.

I saw "stubborn" cows being beaten and squealing pigs chased around the killing floor with electric calipers.

I looked on in utter shock as a cow missed the stun gun and was hoisted fully conscious upside down by its hind leg and cut to pieces, thrashing until its last breath.

Noticing my shock, the foreman remarked: "Who cares? They're going to die anyway."

Because it can cost hundreds of dollars per minutes to stop the conveyor line, animal welfare comes second to profit. Over 150,000 animals are "processed" every hour of every working day in Canada, according to Agriculture Canada.

The picture gets uglier still. En route to slaughter, farm animals may legally spend anywhere from 36 to 72 hours without food, water or rest. They're not even afforded the "luxury" of temperature controlled trucks in extreme summer heat or sub-zero cold. 

Agriculture Canada has estimated that more than 3 million Canadian farm animals die slow and painful deaths en route to slaughter each year.

I've also visited typical Canadian farms. Gone are the days when piglets snorted and roosters strutted their way about the barnyard. Most of today's modernized farms have long, windowless sheds in which animals live like prisoners their entire lives. I have seen chickens crammed four to a cage, nursing pigs separated from their young by iron bars and veal calves confined to crates so narrow they couldn't turn around. Few of these animals ever experience sunlight or fresh air---and most of their natural urges are denied. 

Although it is difficult to face these harsh realities, it is even more difficult to ignore them. Three times a day, you make a decision that not only affects the quality of your life, but the rest of the living world. We hold in our knives and forks the power to change this world.

Consider the words of Albert Einstein: "Nothing will benefit human health and increase the changes for survival of life on Earth as the Evolution to a vegetarian diet."

Bon appetite.

(1196 words)
\bibliography{}
\bibliographystyle{plain}
\end{document}
    
